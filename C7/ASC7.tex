\documentclass[]{article}
\usepackage{lmodern}
\usepackage{amssymb,amsmath}
\usepackage{ifxetex,ifluatex}
\usepackage{fixltx2e} % provides \textsubscript
\ifnum 0\ifxetex 1\fi\ifluatex 1\fi=0 % if pdftex
  \usepackage[T1]{fontenc}
  \usepackage[utf8]{inputenc}
\else % if luatex or xelatex
  \ifxetex
    \usepackage{mathspec}
  \else
    \usepackage{fontspec}
  \fi
  \defaultfontfeatures{Ligatures=TeX,Scale=MatchLowercase}
\fi
% use upquote if available, for straight quotes in verbatim environments
\IfFileExists{upquote.sty}{\usepackage{upquote}}{}
% use microtype if available
\IfFileExists{microtype.sty}{%
\usepackage{microtype}
\UseMicrotypeSet[protrusion]{basicmath} % disable protrusion for tt fonts
}{}
\usepackage[margin=1in]{geometry}
\usepackage{hyperref}
\hypersetup{unicode=true,
            pdftitle={Assignment Regression Model: Automatic or Manual gearbox?},
            pdfauthor={Duy Nguyen},
            pdfborder={0 0 0},
            breaklinks=true}
\urlstyle{same}  % don't use monospace font for urls
\usepackage{color}
\usepackage{fancyvrb}
\newcommand{\VerbBar}{|}
\newcommand{\VERB}{\Verb[commandchars=\\\{\}]}
\DefineVerbatimEnvironment{Highlighting}{Verbatim}{commandchars=\\\{\}}
% Add ',fontsize=\small' for more characters per line
\usepackage{framed}
\definecolor{shadecolor}{RGB}{248,248,248}
\newenvironment{Shaded}{\begin{snugshade}}{\end{snugshade}}
\newcommand{\KeywordTok}[1]{\textcolor[rgb]{0.13,0.29,0.53}{\textbf{#1}}}
\newcommand{\DataTypeTok}[1]{\textcolor[rgb]{0.13,0.29,0.53}{#1}}
\newcommand{\DecValTok}[1]{\textcolor[rgb]{0.00,0.00,0.81}{#1}}
\newcommand{\BaseNTok}[1]{\textcolor[rgb]{0.00,0.00,0.81}{#1}}
\newcommand{\FloatTok}[1]{\textcolor[rgb]{0.00,0.00,0.81}{#1}}
\newcommand{\ConstantTok}[1]{\textcolor[rgb]{0.00,0.00,0.00}{#1}}
\newcommand{\CharTok}[1]{\textcolor[rgb]{0.31,0.60,0.02}{#1}}
\newcommand{\SpecialCharTok}[1]{\textcolor[rgb]{0.00,0.00,0.00}{#1}}
\newcommand{\StringTok}[1]{\textcolor[rgb]{0.31,0.60,0.02}{#1}}
\newcommand{\VerbatimStringTok}[1]{\textcolor[rgb]{0.31,0.60,0.02}{#1}}
\newcommand{\SpecialStringTok}[1]{\textcolor[rgb]{0.31,0.60,0.02}{#1}}
\newcommand{\ImportTok}[1]{#1}
\newcommand{\CommentTok}[1]{\textcolor[rgb]{0.56,0.35,0.01}{\textit{#1}}}
\newcommand{\DocumentationTok}[1]{\textcolor[rgb]{0.56,0.35,0.01}{\textbf{\textit{#1}}}}
\newcommand{\AnnotationTok}[1]{\textcolor[rgb]{0.56,0.35,0.01}{\textbf{\textit{#1}}}}
\newcommand{\CommentVarTok}[1]{\textcolor[rgb]{0.56,0.35,0.01}{\textbf{\textit{#1}}}}
\newcommand{\OtherTok}[1]{\textcolor[rgb]{0.56,0.35,0.01}{#1}}
\newcommand{\FunctionTok}[1]{\textcolor[rgb]{0.00,0.00,0.00}{#1}}
\newcommand{\VariableTok}[1]{\textcolor[rgb]{0.00,0.00,0.00}{#1}}
\newcommand{\ControlFlowTok}[1]{\textcolor[rgb]{0.13,0.29,0.53}{\textbf{#1}}}
\newcommand{\OperatorTok}[1]{\textcolor[rgb]{0.81,0.36,0.00}{\textbf{#1}}}
\newcommand{\BuiltInTok}[1]{#1}
\newcommand{\ExtensionTok}[1]{#1}
\newcommand{\PreprocessorTok}[1]{\textcolor[rgb]{0.56,0.35,0.01}{\textit{#1}}}
\newcommand{\AttributeTok}[1]{\textcolor[rgb]{0.77,0.63,0.00}{#1}}
\newcommand{\RegionMarkerTok}[1]{#1}
\newcommand{\InformationTok}[1]{\textcolor[rgb]{0.56,0.35,0.01}{\textbf{\textit{#1}}}}
\newcommand{\WarningTok}[1]{\textcolor[rgb]{0.56,0.35,0.01}{\textbf{\textit{#1}}}}
\newcommand{\AlertTok}[1]{\textcolor[rgb]{0.94,0.16,0.16}{#1}}
\newcommand{\ErrorTok}[1]{\textcolor[rgb]{0.64,0.00,0.00}{\textbf{#1}}}
\newcommand{\NormalTok}[1]{#1}
\usepackage{graphicx,grffile}
\makeatletter
\def\maxwidth{\ifdim\Gin@nat@width>\linewidth\linewidth\else\Gin@nat@width\fi}
\def\maxheight{\ifdim\Gin@nat@height>\textheight\textheight\else\Gin@nat@height\fi}
\makeatother
% Scale images if necessary, so that they will not overflow the page
% margins by default, and it is still possible to overwrite the defaults
% using explicit options in \includegraphics[width, height, ...]{}
\setkeys{Gin}{width=\maxwidth,height=\maxheight,keepaspectratio}
\IfFileExists{parskip.sty}{%
\usepackage{parskip}
}{% else
\setlength{\parindent}{0pt}
\setlength{\parskip}{6pt plus 2pt minus 1pt}
}
\setlength{\emergencystretch}{3em}  % prevent overfull lines
\providecommand{\tightlist}{%
  \setlength{\itemsep}{0pt}\setlength{\parskip}{0pt}}
\setcounter{secnumdepth}{0}
% Redefines (sub)paragraphs to behave more like sections
\ifx\paragraph\undefined\else
\let\oldparagraph\paragraph
\renewcommand{\paragraph}[1]{\oldparagraph{#1}\mbox{}}
\fi
\ifx\subparagraph\undefined\else
\let\oldsubparagraph\subparagraph
\renewcommand{\subparagraph}[1]{\oldsubparagraph{#1}\mbox{}}
\fi

%%% Use protect on footnotes to avoid problems with footnotes in titles
\let\rmarkdownfootnote\footnote%
\def\footnote{\protect\rmarkdownfootnote}

%%% Change title format to be more compact
\usepackage{titling}

% Create subtitle command for use in maketitle
\newcommand{\subtitle}[1]{
  \posttitle{
    \begin{center}\large#1\end{center}
    }
}

\setlength{\droptitle}{-2em}

  \title{Assignment Regression Model: Automatic or Manual gearbox?}
    \pretitle{\vspace{\droptitle}\centering\huge}
  \posttitle{\par}
    \author{Duy Nguyen}
    \preauthor{\centering\large\emph}
  \postauthor{\par}
    \date{}
    \predate{}\postdate{}
  
\usepackage{booktabs}
\usepackage{longtable}
\usepackage{array}
\usepackage{multirow}
\usepackage[table]{xcolor}
\usepackage{wrapfig}
\usepackage{float}
\usepackage{colortbl}
\usepackage{pdflscape}
\usepackage{tabu}
\usepackage{threeparttable}
\usepackage{threeparttablex}
\usepackage[normalem]{ulem}
\usepackage{makecell}

\begin{document}
\maketitle

\subsection{Introduction}\label{introduction}

This analysis will look at the mtcars dataset and answering two
questions of interest:\\
(1) Is an automatic or manual transmission better for MPG\\
(2) Quantify the MPG difference between automatic and manual
transmissions

\subsection{Exploratory data analysis}\label{exploratory-data-analysis}

The correlations between MPG and other variables show that all variables
have influence on the fuel consumptions. As expected, cyl, disp, hp, wt,
and carb have a negative relationship with MPG. At the fist glance,
tranmission (AM where automatic = 0, manual = 1) seems to be in a
positve relationship with MPG.\\
The violin plot comparing the types of tranmission regarding the fuel
consumption show that, on average, the MANUAL cars consume more fuel
than the AUTOMATIC cars. In fact, this is influenced by other variables
and will be further addressed by linear regression model.

\begin{Shaded}
\begin{Highlighting}[]
\KeywordTok{library}\NormalTok{(datasets)}

\KeywordTok{data}\NormalTok{(mtcars)}
\KeywordTok{kable_styling}\NormalTok{(}\KeywordTok{kable}\NormalTok{(}\KeywordTok{round}\NormalTok{(}\KeywordTok{cor}\NormalTok{(mtcars}\OperatorTok{$}\NormalTok{mpg,mtcars[,}\OperatorTok{-}\DecValTok{1}\NormalTok{]),}\DecValTok{3}\NormalTok{),}\DataTypeTok{caption =} \StringTok{"Correlation between MPG and other variables"}\NormalTok{))}
\end{Highlighting}
\end{Shaded}

\begin{table}

\caption{\label{tab:unnamed-chunk-1}Correlation between MPG and other variables}
\centering
\begin{tabular}[t]{r|r|r|r|r|r|r|r|r|r}
\hline
cyl & disp & hp & drat & wt & qsec & vs & am & gear & carb\\
\hline
-0.852 & -0.848 & -0.776 & 0.681 & -0.868 & 0.419 & 0.664 & 0.6 & 0.48 & -0.551\\
\hline
\end{tabular}
\end{table}

\begin{Shaded}
\begin{Highlighting}[]
\NormalTok{mtcars}\OperatorTok{$}\NormalTok{am <-}\StringTok{ }\KeywordTok{as.factor}\NormalTok{(mtcars}\OperatorTok{$}\NormalTok{am)}
\KeywordTok{levels}\NormalTok{(mtcars}\OperatorTok{$}\NormalTok{am) =}\StringTok{ }\KeywordTok{c}\NormalTok{(}\StringTok{"Automatic"}\NormalTok{,}\StringTok{"Manual"}\NormalTok{)}

\KeywordTok{ggplot}\NormalTok{(mtcars, }\KeywordTok{aes}\NormalTok{(}\DataTypeTok{y=}\NormalTok{mpg, }\DataTypeTok{x =} \KeywordTok{factor}\NormalTok{(am), }\DataTypeTok{fill =} \KeywordTok{factor}\NormalTok{(am), }\DataTypeTok{color =} \KeywordTok{factor}\NormalTok{(am))) }\OperatorTok{+}\StringTok{ }\KeywordTok{geom_violin}\NormalTok{() }\OperatorTok{+}\StringTok{ }\KeywordTok{labs}\NormalTok{(}\DataTypeTok{x=} \StringTok{"Tranmission"}\NormalTok{, }\DataTypeTok{y =} \StringTok{"MPG"}\NormalTok{, }\DataTypeTok{title =} \StringTok{"Comparing automatic and manual regarding fuel consumption (mpg)"}\NormalTok{)}
\end{Highlighting}
\end{Shaded}

\includegraphics{ASC7_files/figure-latex/unnamed-chunk-1-1.pdf}

\subsection{Regression model}\label{regression-model}

\subsubsection{Use tranmission type as the only
predictor}\label{use-tranmission-type-as-the-only-predictor}

To answer the two primary questions, the analysis will, in the first
attempt, model the MPG with only the type of tranmission as a linear
predictor. The summary of the regression shows that with P-value
\textless{} 0.05, we can reject the NULL hypothesis. This means that the
type of tranmission has statiscal significant influence on MPG. The
coefficient of MANUAL is +7.2 which indicates that MANUAL has greater
postive effect on MPG. In other words, MANUAL cars consume more fuel.\\
The explained variance by this linear model, however, is just about 36\%
(indicated by the R squared). This motivates the need for further model
selection considering adjusting the effects of other variables.

\begin{Shaded}
\begin{Highlighting}[]
\NormalTok{fit <-}\StringTok{ }\KeywordTok{lm}\NormalTok{(mpg}\OperatorTok{~}\KeywordTok{factor}\NormalTok{(am), }\DataTypeTok{data =}\NormalTok{ mtcars)}
\NormalTok{(}\KeywordTok{summary}\NormalTok{(fit))}
\end{Highlighting}
\end{Shaded}

\begin{verbatim}
## 
## Call:
## lm(formula = mpg ~ factor(am), data = mtcars)
## 
## Residuals:
##     Min      1Q  Median      3Q     Max 
## -9.3923 -3.0923 -0.2974  3.2439  9.5077 
## 
## Coefficients:
##                  Estimate Std. Error t value Pr(>|t|)    
## (Intercept)        17.147      1.125  15.247 1.13e-15 ***
## factor(am)Manual    7.245      1.764   4.106 0.000285 ***
## ---
## Signif. codes:  0 '***' 0.001 '**' 0.01 '*' 0.05 '.' 0.1 ' ' 1
## 
## Residual standard error: 4.902 on 30 degrees of freedom
## Multiple R-squared:  0.3598, Adjusted R-squared:  0.3385 
## F-statistic: 16.86 on 1 and 30 DF,  p-value: 0.000285
\end{verbatim}

\subsubsection{Model selection by adjusting the effect of other
variables}\label{model-selection-by-adjusting-the-effect-of-other-variables}

Model development could be done in different ways in R, this analysis
will employ the function ``step'' to find the best fitted linear
model.\\
As seen, the model with three variables \textbf{wt}, \textbf{qsec},
\textbf{am} can explain 85\% percent of the variance in the dataset. The
residual plot shows that the residual do not have any pattern. The Q\_Q
plot shows good agreement between theoretical quantiles and the
standardised residuals.

The analysis will choose this as the final model. However, it should be
noted that the model could be further developed to address uncertainty
by considering the interaction between variables.

\begin{Shaded}
\begin{Highlighting}[]
\NormalTok{fit2 <-}\StringTok{ }\KeywordTok{step}\NormalTok{(}\KeywordTok{lm}\NormalTok{(mpg}\OperatorTok{~}\NormalTok{.,}\DataTypeTok{data =}\NormalTok{ mtcars), }\DataTypeTok{trace =} \DecValTok{0}\NormalTok{)}
\KeywordTok{summary}\NormalTok{(fit2)}
\end{Highlighting}
\end{Shaded}

\begin{verbatim}
## 
## Call:
## lm(formula = mpg ~ wt + qsec + am, data = mtcars)
## 
## Residuals:
##     Min      1Q  Median      3Q     Max 
## -3.4811 -1.5555 -0.7257  1.4110  4.6610 
## 
## Coefficients:
##             Estimate Std. Error t value Pr(>|t|)    
## (Intercept)   9.6178     6.9596   1.382 0.177915    
## wt           -3.9165     0.7112  -5.507 6.95e-06 ***
## qsec          1.2259     0.2887   4.247 0.000216 ***
## amManual      2.9358     1.4109   2.081 0.046716 *  
## ---
## Signif. codes:  0 '***' 0.001 '**' 0.01 '*' 0.05 '.' 0.1 ' ' 1
## 
## Residual standard error: 2.459 on 28 degrees of freedom
## Multiple R-squared:  0.8497, Adjusted R-squared:  0.8336 
## F-statistic: 52.75 on 3 and 28 DF,  p-value: 1.21e-11
\end{verbatim}

\begin{Shaded}
\begin{Highlighting}[]
\KeywordTok{par}\NormalTok{(}\DataTypeTok{mfcol =} \KeywordTok{c}\NormalTok{(}\DecValTok{2}\NormalTok{,}\DecValTok{2}\NormalTok{))}
\KeywordTok{plot}\NormalTok{(fit2)}
\end{Highlighting}
\end{Shaded}

\includegraphics{ASC7_files/figure-latex/unnamed-chunk-3-1.pdf}

\subsubsection{Quantifying the difference between MANUAL and AUTOMATIC
gearbox}\label{quantifying-the-difference-between-manual-and-automatic-gearbox}

From the output of the linear model with three regressors \textbf{wt},
\textbf{qsec}, \textbf{am} , it is shown that the MANUAL consume about
\textbf{3} mpg more than the AUTOMATIC cars

\subsection{CONCLUSION}\label{conclusion}

The MPG is unsuprisingly influenced by all variables. However, a linear
model with three variables \textbf{wt}, \textbf{qsec}, \textbf{am} can
explain 85\% of the variance.\\
Answer to the two questions:\\
+ AUTOMATIC is better than MANUAL regarding fuel consumption (MPG)\\
+ AUTOMATIC consumes about 3 mpg less than MANUAL (results extracted
from linear model)


\end{document}
